%%%%%%%%%%%%%%%%%%%%%%%%%%%%%%%%%%%%%%%%%%%%%%%%%%%%%%%%%%%%%%%%%%%%%%%%%%%%%%%%
% Thesis / Project Report
% LaTeX Template
% Version 1.0 (08/04/14)
%
% Author:
% Darshit Shah
% https://github.com/darnir/BPHC-LaTeX-Report-Class
%
% This template is heavily based on the work of Steven Gunn and Sunil Patel
% Steven Gunn
% http://users.ecs.soton.ac.uk/srg/softwaretools/document/templates/
% and
% Sunil Patel
% http://www.sunilpatel.co.uk/thesis-template/
%
% License:
% CC BY-NC-SA 4.0 (http://creativecommons.org/licenses/by-nc-sa/4.0/)
%
% Note:
% Make sure to edit document variables in the Thesis.cls file
%
%%%%%%%%%%%%%%%%%%%%%%%%%%%%%%%%%%%%%%%%%%%%%%%%%%%%%%%%%%%%%%%%%%%%%%%%%%%%%%%%

%-------------------------------------------------------------------------------
%	PACKAGES AND OTHER DOCUMENT CONFIGURATIONS
%-------------------------------------------------------------------------------

\documentclass[11pt, a4paper, oneside]{Thesis} % Paper size, default font size
                                               % and one-sided paper

\graphicspath{{Pictures/}} % Specifies the directory where pictures are stored

\usepackage[square, numbers, comma, sort&compress]{natbib} % Use the natbib
                % reference package - read up on this to edit the reference
                % style; if you want text (e.g. Smith et al., 2012) for the
                % in-text references (instead of numbers), remove 'numbers'

\hypersetup{
    colorlinks = true,
    urlcolor = black,
    linkcolor = blue,
    citecolor = red
} % Sets different colors for different sort of links

\title{\ttitle} % Defines the thesis title - don't touch this

\begin{document}

\frontmatter % Use roman numbering style (i, ii...) for the pre-content pages

\setstretch{1.3} % Line spacing of 1.3

% Define page headers using FancyHdr package and set up for one-sided printing
\fancyhead{} % Clears all page headers and footers
\rhead{\thepage} % Sets the right side header to show the page number
\lhead{} % Clears the left side page header

\pagestyle{fancy} % Finally, use the "fancy" page style to implement the
                  %FancyHdr headers

% New command to make the lines in the title page
\newcommand{\HRule}{\rule{\linewidth}{0.5mm}}

% PDF meta-data
\hypersetup{pdftitle={\ttitle}}
\hypersetup{pdfsubject=\subjectname}
\hypersetup{pdfauthor=\authornames}
\hypersetup{pdfkeywords=\keywordnames}

%-------------------------------------------------------------------------------
%	TITLE PAGE
%- -----------------------------------------------------------------------------

\begin{titlepage}
\begin{center}

\textsc{\LARGE \univname}\\[1.5cm] % University name
\textsc{\Large \doctype}\\[0.5cm] % Thesis type

\HRule \\[0.4cm] % Horizontal line
{\huge \bfseries \ttitle}\\[0.4cm] % Thesis title
\HRule \\[1.5cm] % Horizontal line

\begin{center}
\emph{Author:}\\
\authornames (\idnum)\\
\vspace{0.5cm}
\emph{Supervisors:} \\
\supname \\ % The supervisor's name
\& \\
\cosupname % The co-supervisor's name
\end{center}

\large \textit{A thesis submitted in partial fulfilment of the requirements
    of\\\ccode{} \cname}\\[2cm] % University requirement text

\includegraphics{bits_logo.png}\\ % University/department logo
\UNIVNAME\\
{\large \today}\\[4cm] % Date
\vfill
\end{center}

\end{titlepage}

%-------------------------------------------------------------------------------
%	DECLARATION PAGE
%-------------------------------------------------------------------------------

\Declaration{

\addtocontents{toc}{\vspace{1em}} % Add a gap in the Contents, for aesthetics

I, \authornames, declare that this \doctype{} titled, `\ttitle' and the work
presented in it are my own. I confirm that:

\begin{itemize}
    \item[\tiny{$\blacksquare$}] This work was done wholly or mainly while in
        candidature for a research degree at this University.
    \item[\tiny{$\blacksquare$}] Where any part of this thesis has previously
        been submitted for a degree or any other qualification at this
        University or any other institution, this has been clearly stated.
    \item[\tiny{$\blacksquare$}] Where I have consulted the published work of
        others, this is always clearly attributed.
    \item[\tiny{$\blacksquare$}] Where I have quoted from the work of others,
        the source is always given. With the exception of such quotations, this
        thesis is entirely my own work.
    \item[\tiny{$\blacksquare$}] I have acknowledged all main sources of help.
    \item[\tiny{$\blacksquare$}] Where the thesis is based on work done by
        myself jointly with others, I have made clear exactly what was done by
        others and what I have contributed myself.\\
\end{itemize}

Signed:\\
\rule[1em]{25em}{0.5pt} % This prints a line for the signature

Date:\\
\rule[1em]{25em}{0.5pt} % This prints a line to write the date
}

\clearpage % Start a new page

%-------------------------------------------------------------------------------
%   CERTIFICATE PAGE
%-------------------------------------------------------------------------------

\Certificate{

\addtocontents{toc}{\vspace{1em}}

This is to certify that the thesis entitled, ``\emph{\ttitle}'' and submitted by
\underline{\authornames} ID No. \underline{\idnum} in partial fulfillment of the
requirements of \ccode{} \cname{}embodies the work done by him under my
supervision.\\[2cm]
\begin{minipage}{0.5\textwidth}
\begin{flushleft} \large
\vspace{2cm}
\rule[0.5em]{13em}{0.5pt}\\
\emph{Supervisor}\\
\supname\\
\suppos,\\
\supinst\\
Date:\\
\end{flushleft}
\end{minipage}
\begin{minipage}{0.5\textwidth}
\begin{flushleft} \large
\vspace{2cm}
\rule[0.5em]{13em}{0.5pt}\\
\emph{Co-Supervisor} \\
\cosupname\\
\cosuppos,\\
\cosupinst\\
Date:\\
\end{flushleft}
\end{minipage}\\[3cm]
}
\clearpage

%-------------------------------------------------------------------------------
%	QUOTATION PAGE
%-------------------------------------------------------------------------------

\pagestyle{empty} % No headers or footers for the following pages

\null\vfill % Add some space to move the quote down the page a bit

\textit{``Thanks to my solid academic training, today I can write hundreds of
words on virtually any topic without possessing a shred of information, which is
how I got a good job in journalism."}

\begin{flushright}
Dave Barry
\end{flushright}

% Add some space at the bottom to position the quote just right
\vfill\vfill\vfill\vfill\vfill\vfill\null

\clearpage % Start a new page

%-------------------------------------------------------------------------------
%	ABSTRACT PAGE
%-------------------------------------------------------------------------------

\addtotoc{Abstract} % Add the "Abstract" page entry to the Contents

\abstract{\addtocontents{toc}{\vspace{1em}} % Add a gap in the Contents,
                                            % for aesthetics

The Thesis Abstract is written here (and usually kept to just this page).
The page is kept centered vertically so can expand into the blank space above
the title too\ldots
}

\clearpage % Start a new page

%-------------------------------------------------------------------------------
%	ACKNOWLEDGEMENTS
%-------------------------------------------------------------------------------

\setstretch{1.3} % Reset the line-spacing to 1.3 for body text

\acknowledgements{\addtocontents{toc}{\vspace{1em}} % Add a gap in the Contents.

The acknowledgements and the people to thank go here, don't forget to include
your project advisor\ldots
}
\clearpage % Start a new page

%-------------------------------------------------------------------------------
%	LIST OF CONTENTS/FIGURES/TABLES PAGES
%-------------------------------------------------------------------------------

% The page style headers have been "empty" all this time, now use the "fancy"
% headers as defined before to bring them back
\pagestyle{fancy}

\lhead{\emph{Contents}} % Set the left side page header to "Contents"
\tableofcontents % Write out the Table of Contents

% Set the left side page header to "List of Figures"
\lhead{\emph{List of Figures}}
\listoffigures % Write out the List of Figures

 % Set the left side page header to "List of Tables"
\lhead{\emph{List of Tables}}
\listoftables % Write out the List of Tables

%-------------------------------------------------------------------------------
%	ABBREVIATIONS
%-------------------------------------------------------------------------------

\clearpage % Start a new page

 % Set the line spacing to 1.5, this makes the following tables easier to read
\setstretch{1.5}

\lhead{\emph{Abbreviations}} % Set the left side page header to "Abbreviations"
\listofsymbols{ll} % Include a list of Abbreviations (a table of two columns)
{
\textbf{LAH} & \textbf{L}ist \textbf{A}bbreviations \textbf{H}ere \\
%\textbf{Acronym} & \textbf{W}hat (it) \textbf{S}tands \textbf{F}or \\
}

%-------------------------------------------------------------------------------
%	PHYSICAL CONSTANTS/OTHER DEFINITIONS
%-------------------------------------------------------------------------------

\clearpage % Start a new page

% Set the left side page header to "Physical Constants"
\lhead{\emph{Physical Constants}}

 % Include a list of Physical Constants (a four column table)
\listofconstants{lrcl}
{
Speed of Light & $c$ & $=$ & $2.997\ 924\ 58\times10^{8}\ \mbox{ms}^{-\mbox{s}}$ (exact)\\
% Constant Name & Symbol & = & Constant Value (with units) \\
}

%-------------------------------------------------------------------------------
%	SYMBOLS
%-------------------------------------------------------------------------------

\clearpage % Start a new page

\lhead{\emph{Glossary}} % Set the left side page header to "Symbols"

\listofnomenclature % List the nomenclature. (We use the glossaries package)

%-------------------------------------------------------------------------------
%	DEDICATION
%-------------------------------------------------------------------------------

\setstretch{1.3} % Return the line spacing back to 1.3

\pagestyle{empty} % Page style needs to be empty for this page

% Dedication text
\dedicatory{I would like to dedicate this thesis to my University which taught
me how to deal with all sorts of bureaucracy while doing real work.}

\addtocontents{toc}{\vspace{2em}} % Add a gap in the Contents, for aesthetics

%-------------------------------------------------------------------------------
%	THESIS CONTENT - CHAPTERS
%-------------------------------------------------------------------------------

\mainmatter % Begin numeric (1,2,3...) page numbering

\pagestyle{fancy} % Return the page headers back to the "fancy" style

% Include the chapters of the thesis as separate files from the Chapters folder
% Uncomment the lines as you write the chapters

\input{Chapters/Chapter1}
%\input{Chapters/Chapter2}
%\input{Chapters/Chapter3}
%\input{Chapters/Chapter4}
%\input{Chapters/Chapter5}
%\input{Chapters/Chapter6}
%\input{Chapters/Chapter7}

%-------------------------------------------------------------------------------
%	THESIS CONTENT - APPENDICES
%-------------------------------------------------------------------------------

\addtocontents{toc}{\vspace{2em}} % Add a gap in the Contents, for aesthetics

\appendix % Cue to tell LaTeX that the following 'chapters' are Appendices

% Include the appendices of the thesis as separate files from the Appendices
% folder
% Uncomment the lines as you write the Appendices

\input{Appendices/AppendixA}
%\input{Appendices/AppendixB}
%\input{Appendices/AppendixC}

\addtocontents{toc}{\vspace{2em}} % Add a gap in the Contents, for aesthetics

\backmatter

%-------------------------------------------------------------------------------
%	BIBLIOGRAPHY
%-------------------------------------------------------------------------------

\label{Bibliography}

\lhead{\emph{Bibliography}} % Change the page header to say "Bibliography"

% Use the "unsrtnat" BibTeX style for formatting the Bibliography
\bibliographystyle{ieeetr}

% The references (bibliography) information are stored in the file named
% "Bibliography.bib"
\bibliography{Bibliography}

\end{document}
